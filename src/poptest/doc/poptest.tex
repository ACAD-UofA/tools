\documentclass[aspectratio=1610]{beamer}
\usepackage[authoryear]{natbib}
\usepackage{graphics}
\usetheme{Warsaw}

\begin{document}

\section{$F_{st}$}

\begin{frame}
\frametitle{Weir \& Cockerham (1984) $F_{st}$}
\begin{itemize}
\item Is there greater variance between population groups than within?
\item Power of $F_{st}$ increases when accumulated across multiple loci.
\item Sparse SNP targets makes this problematic.
\item We can estimate uncertainty with jackknife.
\item For each group, $j\in(1..k)$, drop individual $j_i$ from the samples.
	For each remaining group, $m\in(1..k), m!=j$, iteratively drop a second
	individual from the samples.
	Continue for all $i$ in group $j$.
\end{itemize}
\end{frame}

\section{Multiple population test}

\begin{frame}
\frametitle{Null Hypothesis}

\begin{itemize}
\item $H_0$: The population groups are indistinguishable subgroups of a
	single homogeneous population.
\item Consider a single locus.
\item The minor allele frequency for the homogeneous population, $MAF_T$,
	is estimated to be the sum of the minor allele counts for all individuals, $i$,
	divided by the number of alleles counted, $N$.
\begin{equation}
	MAF_T = \frac{1}{2N} \sum_{i} MAC_i
\end{equation}
\item Thus, under the null hypothesis, the probability of drawing a
	minor allele from the population is $MAF_T$.
\end{itemize}
\end{frame}

\begin{frame}
\frametitle{Likelihood Ratio}
\begin{itemize}
\item For population $j$, the likelihood, $L$, of observing $MAC_j$
	minor alleles when sampling $TAC_j$ total alleles, is
\begin{equation}
	L(MAC_j) = Bin(MAC_j, TAC_j, MAF_T)
\end{equation}

Where $Bin(\cdot,\cdot,\cdot)$ is the Binomial probability mass function.

\item Now, any one observation may be unlikely,
	particularly for a large total allele count.
	So we scale by the likelihood of the expected minor allele count.

\begin{equation}
	\frac{L(MAC_j)}{L(E(MAC_j))} = \frac{Bin(MAC_j, TAC_j, MAF_T)}{Bin(E(MAC_j), TAC_j, MAF_T)}
\end{equation}
\end{itemize}

\end{frame}

\begin{frame}
\frametitle{Joint Likelihood Ratio}
\begin{itemize}

\item The joint likelihood ratio for $k$ population groups
	is the product of the likelihood ratios for all the groups.

\begin{equation}
	L = \prod_{j=1}^{k} \frac{L(MAC_j)}{L(E(MAC_j))}
\end{equation}

\end{itemize}
\end{frame}

\begin{frame}
	\frametitle{$E(MAC_j)$}
\begin{itemize}
\item The expected value of the minor allele count, $E(MAC_j)$,
	is an integer close to $MAF_T\times{}TAC_j$
	(we may only observe an integral number of minor alleles).
\item Computationally, we take calculate $L(E(MAC_j))$ as the larger of
	$L(\left\lfloor{MAF_T\times{}TAC_j}\right\rfloor)$ and
	$L(\left\lceil{MAF_T\times{}TAC_j}\right\rceil)$.
	(where $\lfloor{}x\rfloor$ denotes the largest integer not greater than x,
	and $\lceil{}x\rceil$ denotes the smallest integer not less than x)

\end{itemize}

\end{frame}

%\bibliographystyle{plainnat}
%\bibliography{refs}{}

\end{document}
